\thispagestyle{empty}
\باب{دیباچہ}
برقی آلات اور عددی ادوار کے بعد \قول{ مماثل برقیات } میری تیسری کتاب ہے۔یہ کتاب  برقی انجنیئری کی نصابی کتاب  ہے۔امید کی جاتی ہے کہ عملی میدان میں  بھی یہ مددگار ثابت ہو گی۔

کتاب میں تقریباً \عددی{503} اشکال اور \عددی{174} حل شدہ مثال دیے  گئے ہیں۔اس کے علاوہ  مشق کے لئے  \عددی{175} سوالات  بمع جوابات بھی دیے گئے ہیں۔

کتاب \تحریر{Ubuntu} استعمال  کر کے  \تحریر{XeLatex} میں تشکیل دی گئی  اور اس میں   \قول{جمیل نوری نستعلیق } طرز    لکھائی  استعمال کی گئی۔برقیاتی ترسیمات  \تحریر{Octave} جبکہ ادوار \تحریر{gEDA} میں بنائے گئے  ۔ کئی ادوار پر \تحریر{GnuCap} کی مدد سے غور کیا گیا۔میں ان نرم افزار لکھنے والوں کا دل سے شکر گزار ہوں۔طلبہ و طالبات سے گزارش   کی جاتی ہے  کہ وہ اس قسم کے نرم افزار لکھیں یا ان کا ترجمہ علاقائی زبانوں میں کریں۔

کتاب کی تشکیل میں کئی کتابوں کا سہارا لیا گیا۔ ان میں مندرجہ ذیل کا ذکر ضروری ہے۔

{
\begin{otherlanguage}{english}
Electronic Circuits by Schilling-Belove\\
Integrated Electronics by Millman-Halkias\\
Microelectronic Circuits by Sedra-Smith
\end{otherlanguage}
}

اردو اصطلاحات چننے میں درج ذیل لغت سے استفادہ  کیا گیا۔ 

{
\begin{otherlanguage}{english}
http:/\!\!/www.urduenglishdictionary.org\\
http:/\!\!/www.nlpd.gov.pk/lughat
\end{otherlanguage}
}

میں یہاں ان تمام خواتین و حضرات کا شکریہ ادا کرنا چاہتا ہوں جنہوں نے اس کتاب کو مکمل کرنے میں میری مدد کی؛ بالخصوص کامسیٹس میں میرے ساتھی ڈاکٹر عابد حسن مجتبٰے جنہوں نے کتاب کی شکل نکھاری اور میرے شاگرد سید زین عباس، حافظہ مریم اسلم، حرا خان اور  سجّیہ شوکت  جنہوں نے  کتاب کی درستگی کی۔ 

 کتاب کو پہلی مرتبہ بطور نصابی کتاب جن طلباء و طالبات نے پڑھا ان کے نام طلحا ذاہد، عبد اللّہ رضا، عائشہ رباب، سمیا الرحمان، صبح صادق، فیصل پرویز، جبران شبیر اور شاہ زیب علی ہیں۔انہوں نے کتاب کو درست کرنے میں میری مدد کی جس کا میں شکر گزار ہوں۔

 گزارش  کی جاتی ہے کہ   کتاب کو زیادہ سے زیادہ طلبہ و طالبات تک پہنچائیں اور کتاب میں غلطیوں کی نشاندہی میرے  برقیاتی پتہ      پر کریں۔میری  کتابوں کی مکمل \تحریر{XeLatex} معلومات
 
{
\begin{otherlanguage}{english}
https:/\!\!/www.github.com/khalidyousafzai
\end{otherlanguage}
}
سے حاصل کی جا سکتی ہے ، جنہیں آپ مکمل اختیار کے ساتھ استعمال کر سکتے ہیں۔

\vspace{5mm}

{\raggedleft{
خالد خان یوسفزئی

9  نومبر  \سن{2014}}}
