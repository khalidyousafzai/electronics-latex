مساوات \حوالہ{مساوات_ٹرانزسٹر_اسباب_سوارنے_کی_مساوات} کو یہاں اس غرض سے دوبارہ پیش کرتے ہیں۔
\begin{align*}
\Delta I_C \approx \frac{I_{C1}}{\beta_1} \left [\frac{R_B+R_E}{R_B+\left (\beta_2+1 \right )R_E} \right ] \Delta \beta-\frac{1}{R_E} \Delta V_{BE}
\end{align*}
بدترین صورت اس وقت پائی جائے گی جب \عددی{\Delta I_C} کی قیمت زیادہ سے زیادہ ہو۔ایسا \عددی{\Delta V_{BE}=0.6-0.8=\SI{-0.2}{\volt}} اور \عددی{\Delta \beta=150-50=100} پر ہو گا۔ان قیمتوں کو مندرجہ بالا مساوات میں پر کرتے ہیں
\begin{align*}
\Delta I_C &\approx \frac{I_{C1}}{50} \left [\frac{R_B+R_E}{R_B+151 R_E} \right ] \times 100+\frac{0.2}{R_E}\\
&=I_{C1} \left [\frac{R_B+R_E}{75.5 R_E} \right ]+\frac{0.2}{R_E}
\end{align*}
جہاں دوسرے قدم پر تصور کیا گیا ہے کہ دور مساوات \حوالہ{مساوات_ٹرانزسٹر_مخارج_قابو_مزاحمت_کی_شرح} کے تحت بنا ہے  لہٰذا \عددی{ 151 R_E \gg R_B} ہو گا۔یوں قوصین کے نچھلے حصے میں \عددی{R_B} کو نظراندز کیا گیا ہے۔



